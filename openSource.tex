\documentclass[a4paper, 12pt]{article}
\usepackage[utf8]{inputenc}
\usepackage[brazil]{babel}
\usepackage{geometry}
\usepackage{graphicx}
\usepackage{setspace}
\usepackage{lipsum}
\usepackage{listings}
\usepackage{xcolor}
\usepackage{graphicx}
\usepackage{indentfirst}
\renewcommand{\rmdefault}{phv} % Arial as the default font

\setlength{\parindent}{1.25cm}

\geometry{
    left=3cm,
    right=2cm,
    top=3cm,
    bottom=2cm,
}

\begin{document}

\thispagestyle{empty}

\begin{center}
    UNIVERSIDADE FEDERAL DE PERNAMBUCO \\
    CENTRO DE INFORMÁTICA \\
    GRADUAÇÃO CIÊNCIA DA COMPUTAÇÃO

    \vspace{3cm}

    GABRIEL ROCHA \\
    GUSTAVO ARAÚJO  \\
    HEITOR CORDEIRO \\
    IASMIN MARIA GOMES DOS SANTOS \\
    MIGUEL GOMES DE OLIVEIRA\\
    VALTER SANCHES 

    \vspace{4cm}

    Resenha Crítica \\
    Software livre

    
    \vfill

    RECIFE -- PE \\
    2023
\end{center}

\newpage
\thispagestyle{empty}
\begin{center}
    GABRIEL ROCHA \\
    GUSTAVO ARAÚJO  \\
    HEITOR CORDEIRO \\
    IASMIN MARIA GOMES DOS SANTOS \\
    MIGUEL GOMES DE OLIVEIRA\\
    VALTER SANCHES 

    \vspace{6cm}

    Resenha Crítica\\
    Software livre

    \vspace{2cm}

    \hspace*{0.5\textwidth} % Espaço em branco correspondente a 25% da largura da página
    \begin{minipage}{0.5\textwidth} % Define a largura da minipage (50% da largura da página)
        \justifying % Texto justificado
        Resenha crítica referente ao capitulo 18 do Volume 3 do livro "Computação \& Sociedade"  apresentado como requisito do Projeto 1 para à obtenção de aprovação na disciplina de Informatica e Sociedade da Universidade Federal de Pernambuco.

        \vspace{1\baselineskip}
        Prof.: Anderson Rufino, Geber Ramalho, Maria Renata e Veronica Teichrieb
    \end{minipage}

    \vfill

    RECIFE -- PE \\
    2023
\end{center}

\newpage

\section{Introdução e História do Software Livre} 
\paragraph{A história do Software Livre é uma trama fascinante que marcou não apenas avanços técnicos, mas uma verdadeira revolução ideológica na maneira como compreendemos e utilizamos a tecnologia. Desde o pioneirismo de Richard Stallman com o projeto GNU até a consolidação do movimento Open Source, essa narrativa reflete uma transformação profunda na cultura da computação e no acesso ao conhecimento. Contudo, essa trajetória não foi isenta de obstáculos.}
\paragraph{A resistência inicial por parte de empresas e indústrias, a desconfiança na segurança e a incompatibilidade com modelos de negócios estabelecidos foram barreiras significativas para a adoção desses princípios de código aberto. Além disso, a transição do Software Livre para o movimento Open Source, embora tenha facilitado a aceitação no meio corporativo, gerou debates sobre os valores originais desse movimento. Encontrar o equilíbrio entre a liberdade do código aberto e a necessidade prática de segurança e suporte técnico continua sendo um desafio constante para manter esse movimento relevante e impactante na tecnologia e na sociedade.}

\section{Evolução para o Open Source}
\paragraph{A evolução do "Software Livre" para o "Open Source" marcou uma mudança significativa na abordagem do código aberto. Inicialmente baseado em valores éticos e liberdade de acesso ao código, o Software Livre foi impulsionado por uma comunidade comprometida com ideais de autonomia. No entanto, sua ênfase ética muitas vezes dificultava a adoção mais ampla por empresas e desenvolvedores preocupados com eficiência técnica e resultados práticos. Com a transição para o termo Open Source, houve um deslocamento para uma perspectiva mais técnica, valorizando a qualidade do software e o desenvolvimento colaborativo, atraindo assim um público mais amplo, incluindo empresas em busca de eficiência. Essa mudança de terminologia expandiu o alcance do conceito, mas também desencadeou debates sobre a preservação dos valores originais diante da busca por maior aceitação do código aberto.}

\section{Software Livre X Software Proprietário ou Não Livre}
\paragraph{A diferenciação entre software livre e não livre é crucial, delineando a liberdade do usuário e as restrições impostas. No software livre, os usuários desfrutam de acesso, modificação e distribuição do código-fonte, promovendo autonomia e colaboração. Em contrapartida, o software não livre restringe tal acesso e modificações, criando dependência das políticas dos desenvolvedores e limitando a transparência. Essa distinção, porém, não é sempre clara; licenças "livres" podem ter restrições, enquanto softwares "não livres" podem ter acesso gratuito. Ademais, a dependência em padrões fechados dificulta a interoperabilidade, restringindo a liberdade do usuário.}
\paragraph{A separação de software livre e não livre vai além da simples acessibilidade ao código, envolvendo princípios de liberdade e autonomia. Essa distinção complexa desafia uma classificação clara, resultando em debates sobre os limites e implicações práticas desses modelos. A dependência em formatos fechados nos softwares não livres restringe a interoperabilidade, limitando as opções dos usuários. Apesar das diferenciações, licenças "livres" podem conter restrições, enquanto softwares "não livres" às vezes são oferecidos gratuitamente. Essas complexidades perpetuam discussões contínuas sobre a verdadeira essência e as implicações práticas dos modelos de software.}

\section{Vantagens e benefícios do Software Livre}
\paragraph{O Software Livre oferece uma gama de vantagens abrangentes, desde a liberdade do usuário até aspectos práticos de desenvolvimento. A liberdade concedida para acessar, modificar e redistribuir o código-fonte possibilita uma adaptação profunda do software às necessidades individuais ou empresariais, ampliando a flexibilidade e a capacidade de inovar. Além disso, a transparência do código-fonte oferece maior confiabilidade e segurança, pois é revisada por uma comunidade extensa, o que contrasta com a opacidade dos códigos proprietários limitados a equipes internas.}
\paragraph{Outro benefício significativo do Software Livre é a redução de custos, já que geralmente não há despesas de licenciamento envolvidas. Essa característica é atrativa para organizações, permitindo direcionar investimentos para áreas como suporte técnico, personalização ou treinamento, estimulando a inovação. Além disso, a natureza colaborativa do desenvolvimento promove comunidades de desenvolvedores que constantemente aprimoram e inovam, resultando em ciclos de atualização ágeis e eficientes.}
\paragraph{Essas vantagens abrangentes do Software Livre, desde a liberdade de acesso e modificação até a economia de custos e a colaboração na inovação, consolidam a sua posição como uma opção atraente e dinâmica para usuários e organizações em busca de flexibilidade, segurança e eficiência no desenvolvimento e na implementação de software.}

\section{Contribuição em Projetos de Software Livre}
\paragraph{A contribuição para projetos de Software Livre abarca uma série de etapas detalhadas e desafiadoras. Escolher o projeto adequado é crucial, especialmente para iniciantes, pois equilibrar a relevância do projeto com a acessibilidade pode ser complexo. Compreender projetos intricados, configurar o ambiente de desenvolvimento e lidar com questões de compatibilidade entre dependências podem exigir tempo e dedicação, especialmente quando a documentação disponível é limitada. Além disso, encontrar e abordar uma "issue" que corresponda às habilidades e conhecimentos do colaborador pode ser um desafio, assim como produzir código que cumpra os altos padrões do projeto, representando uma etapa crítica do processo.}
\paragraph{Ao longo desse trajeto, cada fase apresenta desafios distintos. Entender um projeto complexo pode demandar tempo considerável, e configurar o ambiente de desenvolvimento pode ser complicado se a documentação for escassa ou se houver problemas de compatibilidade entre as ferramentas necessárias. A seleção de uma "issue" apropriada requer não apenas compreensão técnica, mas também alinhamento com as habilidades e conhecimentos do colaborador, um processo que pode demandar tempo e pesquisa minuciosa. Escrever um código eficiente que atenda aos rigorosos padrões do projeto também é um desafio, especialmente para contribuidores novos, pois o código passará por revisões por pares, exigindo adaptações e aprendizado contínuo.}
\paragraph{A submissão da contribuição é o estágio final, mas não menos desafiador. Enfrentar a revisão por pares e estar aberto a feedback construtivo pode ser estressante, requerendo habilidades para aceitar críticas e realizar ajustes conforme necessário. Essa fase pode representar uma oportunidade valiosa de aprendizado, mas também um momento de pressão para os novos contribuidores na comunidade do Software Livre.}

\section{Críticas e Desafios}
\paragraph{O Software Livre, apesar de suas vantagens significativas, enfrenta críticas relevantes. A ausência de suporte técnico formal é uma desvantagem apontada, especialmente para organizações que dependem de sistemas complexos e críticos. A falta de um suporte dedicado pode resultar em desafios na resolução de problemas e na implementação de soluções específicas, afetando a confiança em sua viabilidade.}
\paragraph{A variedade e liberdade de personalização oferecidas pelo Software Livre podem ser tanto uma vantagem quanto um obstáculo. Para usuários menos técnicos, essa diversidade pode tornar a experiência de uso confusa e desafiadora. Além disso, a ausência de um padrão único pode dificultar a interoperabilidade entre diferentes sistemas, gerando incompatibilidades quando projetos individuais seguem direções de desenvolvimento distintas.}
\paragraph{Enquanto o Software Livre destaca-se por sua liberdade, transparência e custos reduzidos, críticas relacionadas à falta de suporte técnico, complexidade de escolha e questões de padronização e interoperabilidade devem ser consideradas ao avaliar sua adoção em diversos contextos e cenários de uso.}

\section{Conclusão}
\paragraph{A história do Software Livre representa uma jornada fascinante, caracterizada por avanços técnicos e uma revolução na interação com a tecnologia. Olhando para o futuro, a essência do Software Livre permanece na busca pelo equilíbrio entre liberdade e praticidade, onde a adaptação e a inovação se tornam cruciais para manter vivos os princípios de transparência e colaboração. À medida que o movimento avança, encontrar esse equilíbrio torna-se vital, já que o futuro depende da capacidade de se adaptar, inovar e preservar valores fundamentais, impulsionando o progresso tecnológico e fortalecendo a comunidade.}

\end{document}
